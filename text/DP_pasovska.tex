% !TeX spellcheck = cs_CZ
% options:
% thesis=B bachelor's thesis
% thesis=M master's thesis
% czech thesis in Czech language
% slovak thesis in Slovak language
% english thesis in English language
% hidelinks remove colour boxes around hyperlinks
\RequirePackage{color}
\definecolor{gr1}{RGB}{0,153,51}
\documentclass[thesis=M,czech]{FITthesis}[2012/06/26]


\usepackage[utf8]{inputenc} % LaTeX source encoded as UTF-8

\usepackage{graphicx} %graphics files inclusion
\usepackage{amsmath} %advanced maths
\usepackage{amssymb} %additional math symbols
\usepackage{rotating}

\usepackage{dirtree} %directory tree visualisation
% % list of acronyms
% \usepackage[acronym,nonumberlist,toc,numberedsection=autolabel]{glossaries}
% \iflanguage{czech}{\renewcommand*{\acronymname}{Seznam pou{\v z}it{\' y}ch zkratek}}{}
% \makeglossaries
 

\newcommand{\tg}{\mathop{\mathrm{tg}}} %cesky tangens
\newcommand{\cotg}{\mathop{\mathrm{cotg}}} %cesky cotangens

% % % % % % % % % % % % % % % % % % % % % % % % % % % % % % 
% ODTUD DAL VSE ZMENTE
% % % % % % % % % % % % % % % % % % % % % % % % % % % % % % 

\department{Katedra geomatiky     }
\acknowledgements{Studijní program Geodézie a kartografie}

\title{Název}
\authorGN{Petra} %(křestní) jméno (jména) autora
\authorFN{Pasovská} %příjmení autora
\authorWithDegrees{Bc. Petra Pasovská} %jméno autora včetně současných akademických titulů
\supervisor{Ing. Tomáš Janata, Ph.D.}
\acknowledgements{Děkuji Ing. Tomáši Janatovi, Ph.D., za odborné vedení práce a cenné rady, které mi pomohly tuto práci zkompletovat.}
\abstractCS{Abstrakt CZ}
\abstractEN{Abstrakt EN}
\placeForDeclarationOfAuthenticity{V~Praze}
\declarationOfAuthenticityOption{1} %volba Prohlášení (číslo 1-6)
\keywordsCS{klicova slova \newpage}
\keywordsEN{keywords}

\begin{document}

% \newacronym{CVUT}{{\v C}VUT}{{\v C}esk{\' e} vysok{\' e} u{\v c}en{\' i} technick{\' e} v Praze}
% \newacronym{FSv}{FSv}{Fakulta stavebn{\' i}}

\begin{introduction}
Úvod


\end{introduction}

\chapter{Rešerše literatury}


\chapter{Fyzicko geografické informace o oblasti}



\chapter{Kartografické podklady}




\chapter{Kulturní a historické informace o oblasti}


\chapter{Modelování}

\chapter{Diskuze}



\begin{conclusion}
Závěr
\end{conclusion}


\appendix

\chapter{Seznam použitých zkratek}
% \printglossaries
\begin{description}
	\item[GIS] Geografický informační systém

\end{description}



\chapter{Obsah přiloženého CD}

\begin{figure}
	\dirtree{%
		.1 readme.txt\DTcomment{stručný popis obsahu CD}.
		.1 grafy\DTcomment{složka obsahující výsledné grafy}.
		.1 rastry\DTcomment{složka obsahující testované rastry}.
		.1 rozklad.m\DTcomment{skript na výpočet rozkladu RGB barev}.
		.1 LaTex\DTcomment{zdrojová forma práce ve formátu \LaTeX{}}.	
		.1 text\DTcomment{text práce}.
		.2 BP\_Pasovska\_Petra\_2017\DTcomment{text práce v PDF}.	
	}
\end{figure}

\end{document}